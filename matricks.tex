\documentclass[12pt]{article}
\usepackage[noend]{algpseudocode}
\usepackage{algorithm}
\usepackage{amsthm}
\usepackage{amsmath}
\usepackage{amsfonts}
\usepackage{amssymb}
\usepackage[round]{natbib}
\usepackage{multirow}
\usepackage{stfloats}
\usepackage{float}
\usepackage{graphicx}
\usepackage{xcolor}
\usepackage{geometry}
\usepackage{url}
\usepackage[
  bookmarks=false,
  pdfpagelabels=false,
  hyperfootnotes=false,
  hyperindex=false,
  pageanchor=false,
  colorlinks,
  citecolor=blue
]{hyperref}
\usepackage{cleveref}
\usepackage[notref, notcite]{showkeys}

\graphicspath{{../figures/}}
\linespread{1.5}
\bibliographystyle{abbrvnat}
\newcommand\NoDo{\renewcommand\algorithmicdo{}}     % remove "do" from algs

\setlength{\parindent}{0em}
\setlength{\parskip}{.5em}


\newcommand{\vs}{

    \vspace{0.5em}

}

\newcommand{\E}{\mathbb{E}}             % Blackboard E
\newcommand{\R}{\mathbb{R}}             % Blackboard R
\newcommand{\I}{\mathbb{I}}             % Blackboard I
\renewcommand{\P}{\mathbb{P}}           % Blackboard P
\newcommand{\D}{\mathcal{D}}            % Calligraphic D
\newcommand{\M}{\mathcal{M}}            % Calligraphic M
\newcommand{\Q}{\mathcal{Q}}            % Calligraphic Q
\newcommand{\B}{\mathcal{B}}            % Calligraphic B
\renewcommand{\S}{\mathcal{S}}          % Calligraphic S
\renewcommand{\L}{\mathcal{L}}          % Calligraphic L
\newcommand{\Xc}{\mathcal{X}}           % Calligraphic X
\newcommand{\X}{\mathbf{X}}             % Bold X
\newcommand{\KL}{D_\text{KL}}             % KL divergence
\newcommand{\logistic}{\text{logistic}}
\newcommand{\sigmoid}{s}

\newcommand{\argmin}{{\arg\!\min}}      % arg min without space
\newcommand{\argmax}{{\arg\!\max}}      % arg max without space

\newcommand{\red}[1]{{\color{red} #1}}
\newcommand{\blue}[1]{{\color{blue} #1}}

\DeclareMathOperator{\card}{card}
\DeclareMathOperator{\corr}{corr}
\DeclareMathOperator{\cov}{cov}
\DeclareMathOperator{\diag}{diag}
\DeclareMathOperator{\rank}{rank}
\DeclareMathOperator{\tr}{tr}
\DeclareMathOperator{\Var}{Var}


\title{Matricks}
\author{Michael Komodromos}

\renewcommand{\red}[1]{\textcolor{red}{#1}}

\begin{document}
\maketitle

\begin{abstract}
While working with matrices I've picked up some tricks, this document is a collection of them. Very much a WIP.
\end{abstract}

\section{Notation}

\begin{itemize}
    \item $X \in \R^{n \times p}$
    \item $\beta \in \R^p$
    \item $\Sigma \in \R^{p \times p}$ is symmetric and positive-definite with eigenvalues $\lambda_1, \dots, \lambda_p$.
    \item $A = \diag(a_1, \dots, a_n) \in \R{n \times n}$
\end{itemize}



\section{General}

\begin{align}
    \sum_{i=1}^n a_i X_{i \cdot} X_{i \cdot}^\top = &\ X^\top AX
\end{align}

\section{Traces}

\begin{align}
    \| X \beta \|_2^2 = &\ \beta^\top X^\top X \beta = \tr(X^\top X \beta \beta^\top) \\
    \tr(\Sigma \Sigma) = &\ \sum_{i, j} \Sigma_{ij}^2
\end{align}


\section{Cholesky Decomposition}

Write the Cholesky decomposition of $\Sigma = U^\top U = LL^\top$.

\begin{align}
    \det(\Sigma) = &\ \det(U^\top) \det(U) = \left( \prod_{i=1}^p U_{ii} \right)^2 \\
    \tr (\Sigma) = &\ \sum_{i,j} U_{ij}^2 = \sum_{i=1}^p \lambda_i \\
    \Sigma^{-1}  = &\ U^{-1} \left(U^{-1} \right)^\top
\end{align}


\end{document}
